\section{Opleiding}

\cventry{}{Wetenschappen Wiskunde}{EDUGO (De Brug / De Toren)}{}{}{(Secundair Onderwijs)}
\cventry{2016--2019}{Bachelor of Science in de informatica}{Universiteit Gent}{}{}{Geslaagd met onderscheiding}
\cventry{2019--nu}{Master of Science in de informatica}{Universiteit Gent}{}{}{(1ste jaar)}

\section{Werkervaring}

\cventry{zomer 2015}{Vakantiejob bij Tropack}{}{}{}{Inpakken van vrachtwagenonderdelen}
\cventry{zomer 2016}{Vakantiejob bij Carrefour Oostakker}{}{}{}{Rekken vullen in afdeling bazaar / droge voeding}
\cventry{zomer 2017 / zomer 2018}{Vakantiejob bij Lab9K (Digipolis)}{}{}{}{(\url{https://lab9k.gent/}) \\ 
Werken aan IT-projecten in verband met stad Gent \\
Projecten:
\begin{itemize}
\item Decentralized apps op de Ethereum blockchain
  	\begin{itemize}
  		\item ParkCoin \\ 
  			Decentralized app voor registratie / betalen voor parkeren in Gent \\ 
  			(\url{https://github.com/lab9k/ParkCoin})
  	\end{itemize}
		\item Chatbots en Linked Open Data
  	\begin{itemize}
  		\item Gentse Feesten \\ 
  			Chatbot voor evenementen / info over Gentse Feesten gebruik makende van de Linked Open Data \\ 
  			(\url{https://github.com/lab9k/chatbot-visit-gent})
  		\item Citynet Gent \\ 
  			Chatbot bedoeld om vragen te beantwoorden in verband met de besluitvormingen van het Gentse stadsbestuur 
  			gebruik makende van de CityNet API \\ 
  			(\url{https://github.com/lab9k/chatbot-nalantis})
  		\item Toerisme Gent \\ 
  			Chatbot voor toerisme in Gent gebruik makende van de Linked Open Data van Visit Gent \\ 
  			(\url{https://github.com/lab9k/chatbot-toerisme-gent})
  	\end{itemize}
\end{itemize}}
\cventry{zomer 2019}{Vakantiejob bij TomTom}{}{}{}{Ontwikkelen van interne tools voor maps transformations team}