\documentclass[11pt,a4paper,sans]{moderncv}

\usepackage[utf8]{inputenc}
\usepackage[scale=0.75]{geometry}


\moderncvstyle{oldstyle}
\moderncvcolor{grey}
\nopagenumbers{}

\name{Michiel}{Derveeuw}
\address{Absdalestraat 17}{9041 Oostakker}
\phone[mobile]{+32 471 45 1775}
\email{mderveeuw@gmail.com}
\homepage{github.com/mderveeuw}

\begin{document}

\makecvtitle

\section{Opleiding}

\cventry{}{Wetenschappen Wiskunde}{EDUGO (De Brug / De Toren)}{}{}{(Secundair Onderwijs)}
\cventry{2016--2019}{Bachelor of Science in de informatica}{Universiteit Gent}{}{}{}
\cventry{2019--nu}{Master of Science in de informatica}{Universiteit Gent}{}{}{(1ste jaar)}

\section{Werkervaring}

\cventry{zomer 2015}{Vakantiejob bij Tropack}{}{}{}{Inpakken van vrachtwagenonderdelen}
\cventry{zomer 2016}{Vakantiejob bij Carrefour Oostakker}{}{}{}{Afdeling bazaar / droge voeding}
\cventry{zomer 2017 / zomer 2018}{Vakantiejob bij Lab9K}{}{}{}{(\url{https://lab9k.gent/}) \\ 
Werken aan IT-projecten in verband met stad Gent \\
Projecten:
\begin{itemize}
\item Decentralized apps op de Ethereum blockchain
  	\begin{itemize}
  		\item ParkCoin \\ 
  			Decentralized app voor registratie / betalen voor parkeren in Gent \\ 
  			(\url{https://github.com/lab9k/ParkCoin})
  	\end{itemize}
		\item Chatbots en Linked Open Data
  	\begin{itemize}
  		\item Gentse Feesten \\ 
  			Chatbot voor evenementen / info over Gentse Feesten gebruik makende van de Linked Open Data \\ 
  			(\url{https://github.com/lab9k/chatbot-visit-gent})
  		\item Citynet Gent \\ 
  			Chatbot bedoeld om vragen te beantwoorden in verband met de besluitsvormingen van het Gentse stadsbestuur 
  			gebruik makende van de CityNet API \\ 
  			(\url{https://github.com/lab9k/chatbot-nalantis})
  		\item Toerisme Gent \\ 
  			Chatbot voor toerisme in Gent gebruik makende van de Linked Open Data van Visit Gent \\ 
  			(\url{https://github.com/lab9k/chatbot-toerisme-gent})
  	\end{itemize}
\end{itemize}}
\cventry{zomer 2019}{Vakantiejob bij TomTom}{}{}{}{Ontwikkelen van interne tools voor maps transformations team}

\clearpage

\section{Technische ervaring}

\subsection{Programmeertalen}

\cvdoubleitem{Java}{Vloeiend}{Python}{Vloeiend}
\cvdoubleitem{C, C++}{Vloeiend}{Haskell}{Goed}
\cvdoubleitem{JavaScript}{Vloeiend}{R, MATLAB, Maple}{Matig}

\subsection{Web technologieën}

\cvlistdoubleitem{HTML}{CSS}
\cvlistdoubleitem{Angular}{Vue.js}
\cvlistdoubleitem{Node}{Express.js}
\cvlistdoubleitem{Spring Web}{NGINX}

\subsection{Database technologieën}

\cvlistdoubleitem{PostgreSQL}{MySQL}
\cvlistdoubleitem{SQLite}{MongoDB}

\subsection{Tools}

\cvlistdoubleitem{Git}{GitHub}
\cvlistdoubleitem{Docker}{Linux}
\cvlistdoubleitem{Ubuntu}{Arch}
\cvlistdoubleitem{JetBrains IDEs (IntelliJ, Pycharm, Webstorm)}{Visual Studio Code}


\section{Talenkennis}

\cvitem{Nederlands}{Moedertaal}{}
\cvitem{Engels}{Vloeiend}{}
\cvitem{Frans}{Goed}{}
\cvitem{Duits}{Matig}{}

\section{Hobby's / Interesses}

\cvitem{Programmeerwedstrijden}{2 malige deelname aan Vlaamse Programmeerwedstrijd en 2 malige 
deelname aan Google Hash Code}

\end{document}
